\documentclass[conference]{IEEEtran}
\IEEEoverridecommandlockouts
\usepackage{cite}
\usepackage{amsmath,amssymb,amsfonts}
\usepackage{graphicx}
\usepackage{textcomp}
\usepackage{xcolor}
\usepackage{listings} % For code blocks

\def\BibTeX{{\rm B\kern-.05em{\sc i\kern-.025em b}\kern-.08em
    T\kern-.1667em\lower.7ex\hbox{E}\kern-.125emX}}

\begin{document}

\title{A Comparative Study on Machine Learning Models for DDoS Detection in IoT Networks}

\author{\IEEEauthorblockN{Sushil Shakya}
\IEEEauthorblockA{\textit{Bachelors of Information Technology} \\
\textit{Victoria University}\\
Sydney, Australia \\
sushilsayshello@gmail.com}
\and
\IEEEauthorblockN{Supervised by Dr. Robert Abbas}
\IEEEauthorblockA{\textit{Department of Information Technology} \\
\textit{Victoria University}\\
Sydney, Australia \\
robert.abbas@live.vu.edu.au}
}

\maketitle

\begin{abstract}
This paper explores the application of various machine learning models to detect DDoS attacks within IoT environments. Models such as XGBoost, K-Nearest Neighbors (KNN), Stochastic Gradient Descent (SGD), and Naïve Bayes are evaluated for their effectiveness in classifying network traffic as either normal or indicative of a DDoS attack. This comparison highlights the efficacy of these models and their potential to enhance IoT security, especially as IoT device proliferation increases vulnerability to cyber threats.
\end{abstract}

\begin{IEEEkeywords}
DDoS detection, IoT security, machine learning, XGBoost, K-Nearest Neighbors, Stochastic Gradient Descent, Naïve Bayes
\end{IEEEkeywords}

\section{Introduction}

\subsection{Background}
The exponential growth in IoT devices introduces both advancements and security challenges. As IoT devices integrate into various applications, they become susceptible to Distributed Denial of Service (DDoS) attacks, which can overwhelm network resources and render services unavailable \cite{xu2020survey}. Given the scale and inconsistency in security measures across IoT deployments, DDoS vulnerabilities have grown substantially \cite{buczak2016survey}.

\subsection{Motivation}
Traditional detection methods reliant on fixed rules or signatures are insufficient for the evolving nature of cyber-attacks, especially within IoT. Machine learning (ML) offers adaptive and robust mechanisms capable of real-time anomaly detection without prior knowledge of attack signatures \cite{zargar2013survey}. This paper aims to evaluate and contrast ML models to determine the optimal choice for DDoS detection in IoT environments.

\subsection{Contribution}
This study compares the performance of XGBoost, KNN, SGD, and Naïve Bayes in identifying DDoS patterns in network traffic. By analyzing these models, we contribute insights into machine learning applications for enhancing IoT security. 

\section{Literature Review}

\subsection{Traditional DDoS Detection Techniques}
Conventional DDoS detection relies heavily on fixed rules or signature-based identification, which fails to adapt to new attack strategies in dynamic network environments \cite{mirkovic2004taxonomy}.

\subsection{Machine Learning in DDoS Detection}
ML-based approaches categorize techniques into supervised, unsupervised, and semi-supervised learning:
\begin{itemize}
    \item \textbf{Supervised Learning}: Models are trained with labeled datasets containing normal and attack traffic. This approach can achieve high accuracy but requires constant updates \cite{sommer2010closed}.
    \item \textbf{Unsupervised Learning}: Identifies attacks by observing deviations from normal patterns, useful for detecting zero-day attacks but prone to false positives \cite{divekar2018benchmarking}.
\end{itemize}

\section{Methodology}

\subsection{Dataset Description}
A labeled dataset of IoT network traffic was used, including features like packet size, rate, and protocol type. The dataset underwent preprocessing for data cleaning, feature selection, and normalization, employing Min-Max scaling:
\begin{equation}
    x' = \frac{x - \min(x)}{\max(x) - \min(x)}
\end{equation}

\begin{figure}[htbp]
\centerline{\includegraphics[width=\linewidth]{methodology_image.png}}
\caption{Sample architecture of ML-based DDoS detection in IoT}
\label{fig:methodology}
\end{figure}

\subsection{Model Descriptions}
\begin{itemize}
    \item \textbf{XGBoost}: A gradient-boosted decision tree algorithm known for handling complex datasets and minimizing overfitting. Objective function:
    \begin{equation}
        \mathcal{L} = \sum_{i=1}^{n} l(y_i, \hat{y}_i) + \sum_{k=1}^{K} \Omega(f_k)
    \end{equation}

    \item \textbf{KNN}: An instance-based algorithm that assigns a sample the majority class among its \( k \) nearest neighbors. Distance metric:
    \begin{equation}
        d(x, y) = \sqrt{\sum_{i=1}^{n} (x_i - y_i)^2}
    \end{equation}

    \item \textbf{SGD}: An optimization algorithm that iteratively updates parameters. Update rule:
    \begin{equation}
        \theta := \theta - \eta \nabla_\theta J(\theta)
    \end{equation}

    \item \textbf{Naïve Bayes}: Probabilistic classifier based on Bayes' theorem, assuming feature independence:
    \begin{equation}
        P(C_k | x) = \frac{P(x | C_k) \cdot P(C_k)}{P(x)}
    \end{equation}
\end{itemize}

\subsection{Evaluation Metrics}
Metrics used to assess the models include:
\begin{itemize}
    \item \textbf{Accuracy}: \( \text{Accuracy} = \frac{\text{TP + TN}}{\text{TP + TN + FP + FN}} \)
    \item \textbf{Precision}: \( \text{Precision} = \frac{\text{TP}}{\text{TP + FP}} \)
    \item \textbf{Recall}: \( \text{Recall} = \frac{\text{TP}}{\text{TP + FN}} \)
    \item \textbf{F1 Score}: \( \text{F1 Score} = 2 \cdot \frac{\text{Precision} \cdot \text{Recall}}{\text{Precision + Recall}} \)
\end{itemize}

\section{Results}

\begin{table}[htbp]
\caption{Performance Metrics of Machine Learning Models}
\begin{center}
\begin{tabular}{|c|c|c|c|c|}
\hline
Model & Accuracy & Precision & Recall & F1 Score \\
\hline
XGBoost & 99.82\% & 99.80\% & 99.85\% & 99.82\% \\
KNN & 99.56\% & 99.50\% & 99.60\% & 99.55\% \\
SGD & 98.25\% & 98.10\% & 98.30\% & 98.25\% \\
Naïve Bayes & 91.09\% & 90.70\% & 91.10\% & 90.90\% \\
\hline
\end{tabular}
\label{table:performance}
\end{center}
\end{table}

\begin{figure}[htbp]
\centerline{\includegraphics[width=\linewidth]{results_chart.png}}
\caption{Performance Comparison of ML Models}
\label{fig:results_chart}
\end{figure}

\section{Discussion}
XGBoost yielded superior accuracy and reliability in detecting DDoS attacks, supported by its handling of large-scale data. KNN and SGD performed closely but with limitations in high-dimensional data. Naïve Bayes, while efficient, exhibited higher misclassification rates.

\section{Conclusion}
Our study demonstrates the efficacy of ML in enhancing IoT network security, with XGBoost showing the highest potential for DDoS detection. Future work should explore hybrid models and test scalability within real-world IoT infrastructures.

\section*{Acknowledgment}
The authors extend gratitude to Dr. Robert Abbas for his insightful guidance throughout this research.

\begin{thebibliography}{00}
\bibitem{mirkovic2004taxonomy} J. Mirkovic and P. Reiher, ``A taxonomy of DDoS attack and DDoS defense mechanisms,'' \emph{ACM SIGCOMM Computer Communication Review}, vol. 34, no. 2, pp. 39--53, Apr. 2004.
\bibitem{zargar2013survey} S. T. Zargar, J. Joshi, and D. Tipper, ``A survey of defense mechanisms against distributed denial of service (DDoS) flooding attacks,'' \emph{IEEE Communications Surveys \& Tutorials}, vol. 15, no. 4, pp. 2046--2069, Fourthquarter 2013.
\bibitem{xu2020survey} L. Xu, Y. Tian, and S. Sengupta, ``A comprehensive survey of DDoS attacks and defense mechanisms in the IoT,'' \emph{IEEE Communications Surveys \& Tutorials}, vol. 22, no. 3, pp. 1530--1567, thirdquarter 2020.
\bibitem{buczak2016survey} A. L. Buczak and E. Guven, ``A survey of data mining and machine learning methods for cybersecurity intrusion detection,'' \emph{IEEE Communications Surveys \& Tutorials}, vol. 18, no. 2, pp. 1153--1176, secondquarter 2016.
\bibitem{sommer2010closed} R. Sommer and V. Paxson, ``Outside the closed world: On using machine learning for network intrusion detection,'' in \emph{Proc. IEEE Symposium on Security and Privacy}, Oakland, CA, USA, May 2010, pp. 305--316.
\bibitem{divekar2018benchmarking} A. Divekar, A. Parekh, J. Savla, A. Mishra, and M. Shirole, ``Benchmarking framework for myopic intrusion detection systems,'' \emph{Security and Communication Networks}, vol. 2018, Article ID 3724931, 18 pages, 2018. doi:10.1155/2018/3724931.
\end{thebibliography}

\end{document}

